\addchap{Abstract}
\label{ch:abstract}

\emph{The glass transition is often thought as decoupled from any structural change. I show in this thesis that two types of local order can be detected in a simple experimental glass former. This order increases when approaching the glass transition and is spatially correlated with the dynamic heterogeneities of the supercooled liquid.}


\paragraph{}
Approaching the glass transition temperature $T_g$, the temperature dependence of the viscosity or the relaxation time follows at least the Arrhenius law. Glass formers that have a super-Arrhenius behaviour are called fragile, as opposed to strong. It was suggested recently that the fragility of glass formers could be related to the frustration against crystallisation. The glass transition temperature is not a well-defined thermodynamic quantity; it depends on the cooling rate and on the time scale accessible to the experimentalist. Rather than $T_g$, some theories use a temperature $T_0$, which corresponds to the temperature where the viscosity and structural relaxation time diverge.

Nothing spectacular happens in the positional order of the system at the glass transition : no breaking of symmetry, no drastic structural change. In critical phenomena a static correlation length (the characteristic size of the critical fluctuations) diverges, which is responsible for the dramatic slowing down of the system. No diverging static correlation length could be found for decades in glass forming liquids.

The glass is a non-ergodic state of matter where the system is stuck in one configuration, unable to rearrange. By contrast, a supercooled liquid is ergodic even if metastable with respect to the crystal. Supercooled liquids show characteristic dynamical signatures, which include a plateau in the mean square displacement of the particles, a two step relaxation process, non-gaussianity of the dynamics and stretching of the decay of correlation functions.

For more than a decade, it is known that the non-gaussianity and the stretching are due to dynamic heterogeneities. At a given time, some regions of the supercooled liquid are fast and others are slow. Dynamic heterogeneities are transient and temporally fluctuating. Their size could be characterised using a four-point density correlation function and this dynamic length scale diverges toward the glass transition.

Is then the glass transition a purely dynamic phenomenon? To answer this question Widmer-Cooper and Harrowell ran many simulations from the same initial configuration but with different initial velocities. In this way, they obtained the propensity to move of the different places of this initial configuration independently of the initial dynamic. This propensity was heterogeneous, so some places have a higher probability to move than others. It was then shown by Berthier and Jack that the propensity has a predictive power on the actual dynamic of a given run, but only on medium range scale. Propensity does not predict the dynamic of a given particle but of a mesoscopic region. Nevertheless these works show that dynamic heterogeneities do have a static structural cause – at least in the most common systems.

\paragraph{}

We are faced with an apparent contradiction: supercooled liquids seem to be amorphous but have some sort of transient local or medium range structure that causes dynamic heterogeneities. This contradiction resolves when considering a structural order criteria catching the local symmetry rather than the long range positional order. An ordered particle is then a particle with a highly symmetric neighbourhood, even if this symmetry is distorted or non-existent over a longer range.

This type of structural order is not obtainable experimentally by usual scattering techniques. However if one has the coordinates of the individual particles, like in a simulation, one can use an order parameter taking into account the angles between the bonds linking particles. For example, the bond orientational order developed by Steinhardt and co-workers is used to study quasi-crystals, crystals and crystallisation. Steinhardt bond orientational order is a tensorial order parameter that can be defined for all $\ell$-fold symmetry with even $\ell$. The rotational invariants of the tensor are called $q_\ell$ and $w_\ell$. $q_\ell$ describes the strength of the $\ell$-fold symmetry. For example icosahedra and crystals (BCC, FCC, HCP) have a strong 6-fold symmetry and thus high values of $q_6$. $w_\ell$ takes a precise value for a given symmetry group. For example the icosahedral symmetry gives $w_6= -\frac{11}{\sqrt{4199}} \sim -0.169$, whereas the values of all crystals symmetry groups collapse to almost zero.

At finite temperatures, the structures are distorted by vibration. This makes the distributions of the $q_\ell$ and $w_\ell$ broad, noisy and overlapping. One can characterise a sample as a mixture of two or three structures, but it is impossible to identify the structure of a single particle. If one takes into account the symmetry of the second shell around a particle and not only the nearest neighbours, the distributions of the coarse-grained $Q_\ell$ and $W_\ell$ are more robust to thermal fluctuations and less noisy. Structures with a small amount of periodicity (crystal-like) can be resolved at the particle level. Moreover, the signal of the non-periodic structures like icosahedron shrinks to zero. The coarse-grained $Q_6$ is thus a very good indicator of local crystallinity, and local crystallinity only.

Using this technique to analyse simulations of particles close to hard spheres, Kawasaki and Tanaka showed recently that the fluctuations of the local crystalline order are closely related to dynamic heterogeneities. Moreover, the length scale of these fluctuations show an Ising-like power-law divergence toward the glass transition point. These results suggest a far more direct link than thought before between the glass transition and critical phenomena. Indeed, the glass transition may be a new type of critical phenomenon where a structural order parameter is directly linked to slowness. Moreover, this structural ordering accompanies little change in density, which explains why it has not been detected by the static structure factor so far.

\paragraph{}

Only a few experimental systems allow access to the individual coordinates of the particles of a supercooled liquid and thus to the bond orientational order. Recently such analysis was performed on two dimensional driven granular matter. To investigate a three dimensional system, we had to switch to colloids. Confocal microscopy enables us to image colloids in three dimensions. A colloidal suspension is a realistic yet simple system that is able to model many features of condensed matter physics. To our knowledge, colloids were the best choice to investigate experimentally the influence of local order on the glassy dynamics.

Our colloids were designed to behave like hard spheres, one of the most well-studied systems of statistical physics. In particular, it was shown that a system of identical hard spheres has a well-defined freezing transition upon increasing density, driven by purely entropic effects. Hard sphere-like colloids exhibit this transition. Crystallisation can be frustrated by some amount of size polydispersity of the particles, allowing supercooling and a glass transition.

We tracked our colloids to obtain the coordinates of each particle at each time step and following them in time. In order to track tens of thousands particles in a reliable way over extended periods of time, we had to develop a new tracking software with extended capabilities and a gain of speed that is between one and two orders of magnitude compared to existing implementations.

From the coordinates linked into trajectories, we were able to compute dynamic quantities (intermediate scattering function, mean square displacement, etc.), both globally and locally. We confirmed that our system exhibits the dynamical signature of super-cooled liquids, including the dynamical heterogeneities.

We analysed the local structure of our samples, using both the coarse grained and non-coarse grained version of Steinhardt bond orientational order parameter. Surprisingly, the non-coarse-grained $w_6$ revealed a strong tendency toward icosahedral order. Particles interacting with an attractive potential have a preferred bond length and naturally form highly symmetric structures at low enough temperatures. For example 13 Lennard-Jones particles in isolation form an icosahedron. For purely repulsive particles, only packing and entropy effects can act to promote local order. Fragments of icosahedra have been detected before in hard-sphere systems, but it was at very high volume fractions, even higher than the glass transition. Icosahedral order was not expected in this proportion in moderately supercooled liquid of hard spheres.

This result was confirmed by re-analysing Kawasaki's simulation data of a similar polydisperse hard-sphere-like system. Icosahedra are detected even in simulations of truly monodisperse systems. The short lived supercooled state contains icosahedra in a proportion similar to polydisperse systems. They are not an effect of polydispersity but may be stabilised by it. The presence of icosahedral clusters is an intrinsic source of frustration to crystallisation in the system. This may explain why hard spheres show some amount of fragility even at zero polydispersity.

We could also confirm experimentally the results of Kawasaki about the local crystalline order parameters $Q_6$. We also detect transient medium range crystalline order reminiscent of critical fluctuations. The high $Q_6$ regions have a strong 6-fold symmetry, but should not be confused with crystal nuclei. Their density is basically the same as the disordered parts. Moreover, they lose their periodicity after the second shell. We could extract a characteristic decay length $\xi_6$ from its spatial correlation function. the volume fraction dependence of $\xi_6$ agrees quantitatively with simulations, diverging toward $T_0$ in a Ising-like fashion.

We could also observe heterogeneous nucleation of crystal at a flat wall. The wall induces layering that promote hexatic plane formation. Even if the equilibrium structure should be Face Centered Cubic crystal (FCC), some planes misalign, forming a random hexagonal compact structure (RHCP). This effect is said to be due to the low free energy difference between FCC and HCP in hard spheres. We could observe this phenomenon during crystal growth and monitor the crystal structures at the particle level using the coarse grain $W_4$. We were also able to see how growing crystals interact with the symmetrically-inconsistent icosahedra.

Despite an extensive exploration of the various parameters exposed by Steinhardt bond orientational order, we could not find other relevant structures in our experimental system or in the simulations. BCC order is absent and dodecahedra are extremely short-lived. The structure of hard sphere systems can then be summarised by a map in the $(w_6, Q_6)$ plane. $Q_6$ monitors the tendency toward crystallisation (HCP or FCC crystal), whereas $w_6$ follows the formation of icosahedra. We showed that both tendencies are present in hard spheres supercooled liquids and are mutually exclusive.

With this simplified map, we were able to correlate the dynamics of individual particles to their local symmetry. Statistically, ordered particles are slower than the average and rearrange less frequently. On larger length scales, we confirm a mutual exclusion between theses two types of ordered clusters and the fast dynamics regions.

\paragraph{}

Our results suggest that the dynamical arrest of glass is due to the presence of two types of incompatible local order in the system. One of the two is related to the (crystalline) ground state of the system. The other frustrates crystallisation and is locally stable. The local crystalline order parameter presents critical-like fluctuations growing toward the glass transition. The diverging length scale of theses fluctuations explains the global slowing down of the dynamics. Without the very stable local patches of the second type of order,  the crystal would just fill space.