\chapter{Conclusion}

Using confocal microscopy of colloids, we were able to observe the behaviour of hard-sphere supercooled liquid at the particle level. We investigated the slow dynamics on a time scale spanning nearly four decades. Our system exhibits the characteristic signature of supercooled liquids including two-step relaxation, non-gaussianity and heterogeneity of the dynamics. The characteristic length of the dynamic heterogeneities was found to follow a power law diverging toward the ideal glass transition.

The local structures of the system were investigated into detail to discover two main structures that extend over medium range: the crystal-like bond order forming clusters and the icosahedral order forming a network. This second type of order is present even at low volume fractions and between growing crystal nuclei. It is intrinsic to the liquid and stabilized at the local level by free volume effects.

However, contrary to the common viewpoint\citep{tarjus2005fba} and observations in other model systems~\citep{Dzugutov2002}, the icosahedral order is not the dominant one near the glass transition. In hard spheres, the characteristic size of the fluctuations of the crystal-like bond order follows the same power law divergence as the dynamic correlation length and the crystal-like bond ordered regions are the slowest and longest lived structures of the system: they are responsible for the dynamic arrest.

The role of the icosahedral order is more important as a source of frustration against the crystal-like bond order rather than a structure causing the slow dynamics. We showed that the crystal-like bond order is the precursor of crystallisation. Thus, preventing the formation of crystal-like bond ordered regions implies both preventing crystallisation and delaying the slowing down of the system. Therefore, the locally favoured icosahedral order is responsible for both the avoidance of crystallisation leading to supercooling and the fragility of the glass former.